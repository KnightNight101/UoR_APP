\documentclass[12pt,a4paper]{report}

% Packages
\usepackage[utf8]{inputenc}
\usepackage{graphicx}
\usepackage{amsmath,amssymb}
\usepackage{hyperref}
\usepackage{geometry}
\usepackage{titlesec}
\usepackage{fancyhdr}
\usepackage{setspace}

% Page settings
\geometry{margin=1in}
\setlength{\parskip}{0.8em}
\setlength{\parindent}{0pt}
%\doublespacing 

% Header and Footer
\pagestyle{fancy}
\fancyhf{}
\rhead{\thepage}
\lhead{ML-Powered Agile PM App}

% Title
\title{\textbf{An LLM-Powered Application for Agile Project Management}}
\author{Nithin Gandhi Simanand \\ Supervisor: Pat Parslow \\ MSc Data Science and Advanced Computing}
\date{\today}

\begin{document}

\maketitle

\tableofcontents
\newpage

% 1. Introduction
\chapter{Introduction}  % ~800 words
\section{Problem Statement and Motivation}
Enterprise-scale organizations routinely operate across a complex matrix of interdependent projects, distributed teams, and extensive resource portfolios. To maintain operational coherence and deliver value efficiently, these firms typically rely on structured project management frameworks such as PRINCE2, PMBOK, or Agile-based hybrids. While such methodologies have demonstrated clear benefits in improving project visibility, governance, and stakeholder alignment, they remain susceptible to inherent inefficiencies. These stem largely from the rigidities of hierarchical communication, bureaucratic inertia, and the sheer scale of coordination required in large corporate environments.

Despite adherence to best practices, many firms continue to experience avoidable project delays, resource misallocations, and suboptimal decision-making cycles. A significant portion of these inefficiencies can be traced to human limitations in managing high-dimensional data, repetitive administrative tasks, and the cognitive overload associated with large-scale project orchestration. This presents a critical bottleneck: even with mature frameworks in place, enterprise project management often lacks the real-time adaptability and computational agility required to fully optimize performance at scale.

The emergence of Large Language Models (LLMs) and Machine Learning (ML) systems offers a transformative opportunity to address these structural inefficiencies. These technologies excel at pattern recognition, predictive analysis, and automating low-level cognitive functions — attributes that align closely with the pain points of modern project management. By offloading repetitive, computation-heavy, and data-intensive components of project workflows to intelligent systems, organizations can enable human stakeholders to concentrate on strategic, creative, and high-context tasks where human judgment is irreplaceable.

This research is motivated by the growing imperative to integrate intelligent automation into enterprise project management pipelines — not as a replacement for human oversight, but as a complementary force multiplier. The central problem this study seeks to explore is how to effectively design, implement, and validate LLM- and ML-driven solutions that can reduce latency, enhance decision quality, and streamline resource utilization in large-scale organizational project environments. The challenge lies not merely in the application of these technologies, but in aligning them with existing frameworks, managing their integration within corporate governance structures, and ensuring their outputs remain interpretable, auditable, and aligned with business objectives.

\section{Project Objectives}


% 2. Background & Literature Review
\chapter{Background and Literature Review}  % ~2000 words
\section{Agile Methodology: Overview and Key Concepts}
\section{Existing Project Management Tools}
\section{Applications of ML/LLMs in Project Management}
\section{LLM Technologies: GPT, BERT, and Others}
\section{Gaps in Current Research and Tools}

% 3. Methodology
\chapter{Methodology}  % ~1500 words
\section{Requirements Gathering}
\section{Design Methodology}
\section{Data Sources and Preprocessing}
\section{LLM Selection and Integration Strategy}
\section{System Architecture Overview}

% 4. System Design & Implementation
\chapter{System Design and Implementation}  % ~2000 words
\section{System Architecture}
\section{Frontend and Backend Design}
\section{Key Features and Functionalities}
\section{LLM Integration and Prompt Engineering}
\section{Security and Ethical Considerations}

% 5. Evaluation & Testing
\chapter{Evaluation and Testing}  % ~1500 words
\section{Evaluation Criteria}
\section{Functional Testing}
\section{LLM Response Evaluation}
\section{User Testing and Feedback}
\section{Limitations in Testing}

% 6. Discussion
\chapter{Discussion}  % ~1000 words
\section{Analysis of Results}
\section{Challenges Faced}
\section{Agile Methodology Reflection}
\section{Model Performance vs Expectations}

% 7. Conclusion & Future Work
\chapter{Conclusion and Future Work}  % ~700 words
\section{Summary of Contributions}
\section{Applications and Impact}
\section{Limitations}
\section{Future Improvements}

% Appendices (not included in word count)
\appendix
\chapter{Screenshots and Interface Designs}
\chapter{Test Logs and Scripts}
\chapter{User Feedback and Survey Responses}

% References
\bibliographystyle{IEEEtran} % Or change to your required style
\bibliography{references}  % Add your .bib file later

\end{document}
