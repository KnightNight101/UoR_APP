\documentclass{report}
% Packages
\usepackage[utf8]{inputenc}
\usepackage{graphicx}
\usepackage{amsmath,amssymb}
\usepackage{hyperref}
\usepackage{geometry}
\usepackage{titlesec}
\usepackage{fancyhdr}
\usepackage{setspace}
\usepackage{csquotes}
\usepackage[style=authoryear-ibid,backend=biber]{biblatex}
\addbibresource{UoR_Masters_Project.biblatex}

% Page settings
\geometry{margin=1in}
\setlength{\parskip}{0.8em}
\setlength{\parindent}{0pt}
\doublespacing% Header and Footer
\pagestyle{fancy}
\fancyhf{}
\rhead{\thepage}
\lhead{ML-Powered Agile PM App}

% Title
\title{\textbf{An LLM-Powered Application for Agile Project Management}}
\author{Nithin Gandhi Simanand \\ Supervisor: Pat Parslow \\ MSc Data Science and Advanced Computing}
\date{\today}

\begin{document}

\maketitle

\tableofcontents
\newpage

% 1. Introduction
\chapter{Introduction}  % ~800 words
\section{Problem Statement and Motivation}

Enterprise-scale organisations routinely operate across a complex matrix of interdependent projects, distributed teams, and extensive resource portfolios. To maintain operational coherence and deliver value efficiently, these firms typically rely on structured project management frameworks such as PRINCE2, PMBOK, or Agile-based hybrids. While such methodologies have demonstrated clear benefits in improving project visibility, governance, and stakeholder alignment, they remain susceptible to inherent inefficiencies. These stem largely from the rigidities of hierarchical communication, bureaucratic inertia, and the sheer scale of coordination required in large corporate environments \parencite{pricaEnhancingProjectEfficiency2025}.

Despite adherence to best practices, many firms continue to experience avoidable project delays, resource misallocations, and suboptimal decision-making cycles \parencite{mankinsTurningGreatStrategy}. A significant portion of these inefficiencies can be traced to human limitations in managing high-dimensional data, repetitive administrative tasks, and the cognitive overload associated with large-scale project orchestration. This presents a critical bottleneck: even with mature frameworks in place, enterprise project management often lacks the real-time adaptability and computational agility required to fully optimize performance at scale.

The emergence of Large Language Models (LLMs) and Machine Learning (ML) systems offers a transformative opportunity to address these structural inefficiencies. These technologies excel at pattern recognition, predictive analysis, and automating low-level cognitive functions, all attributes that align closely with the pain points of modern project management. By offloading repetitive, computation-heavy, and data-intensive components of project workflows to intelligent systems, organizations can enable human stakeholders to concentrate on strategic, creative, and high-context tasks where human judgment is irreplaceable.

However, due to the ever-growing importance of data protection, firms are increasingly hesitant to integrate cloud-based AI providers into their workflows. Concerns surrounding the ambiguity of how data is processed, where it is transmitted, and which datasets were used to train these models have created a significant barrier to adoption. While some organizations attempt to mitigate this risk by deploying locally hosted LLMs, these models typically function as open-ended chatbots. Consequently, the quality and relevance of their outputs are highly dependent on the user's ability to craft precise, context-aware prompts. Given the diversity in LLM interfaces, capabilities, and task specialization, employees without training in prompt engineering or a deep understanding of model behavior often struggle to extract meaningful value — effectively neutralizing the potential gains these tools could offer.

This research is motivated by the need to bridge the gap between the theoretical capabilities of intelligent automation and their practical usability in real-world enterprise settings. The core problem lies not only in technical implementation, but also in aligning these systems with organizational workflows, ensuring security and compliance, and designing interfaces and processes that make advanced tools accessible to non-expert users. The goal is to explore how LLM- and ML-driven solutions can be securely, effectively, and intuitively integrated into large-scale project environments — enabling genuine improvements in efficiency, responsiveness, and decision quality without compromising data integrity or overwhelming the workforce.


\section{Project Objectives}

There are four main objectives for this project:
\subsection{Develop a Multi-Dimensional Sprint Planner}
The first goal is to design an intelligent sprint planning system that automates the prioritization and scheduling of project tasks. This planner considers multiple dynamic inputs, including team member availability, skill sets, task dependencies, organizational constraints, and regulatory and compliance related deadlines. The system is built to continuously ingest real-time updates, allowing it to refine project roadmaps dynamically as requirements shift. By embedding adaptive logic into the planner, the tool ensures that large-scale delivery remains aligned with strategic goals and operational timelines.
\subsection{Implement an Eisenhower Matrix-Based To-Do List Generator}
To enhance daily execution at the individual level, the system generates personalized task lists using the Eisenhower Matrix framework. This model categorizes tasks based on urgency and importance, helping users focus on high-impact work while deferring or delegating less critical activities. These lists are continuously updated based on project context and user input, supporting focused, high-leverage productivity without manual task sorting.
\subsection{Integrate a Version Control System with LLM-Generated Commit Summaries}
The third objective is to ensure knowledge continuity and reduce the cognitive cost of context switching by embedding a version control system (VCS) into the platform. The VCS tracks project artifacts, decisions, and revisions, while an LLM generates automatic commit summaries that document key changes in clear, concise language. These summaries can be aggregated into progress reports, streamlining communication with internal stakeholders and external clients. Additionally, the system surfaces relevant historical context during task transitions, reducing ramp-up time and minimizing productivity losses across team handovers or project shifts.
\subsection{Ensure Secure, Containerized Local Deployment}
Finally, the project emphasizes data security and deployment flexibility by containerizing the entire application for on-premise use. This architecture ensures that all data processing occurs within the organization’s infrastructure, mitigating the risks associated with cloud-based AI platforms. The modular software design also supports plug-and-play integration of different LLMs for specialized tasks, allowing components to be updated independently without disrupting the user experience. By abstracting LLM interactions behind clearly defined workflows, the system eliminates the need for prompt engineering, making advanced AI capabilities accessible to non-technical users.

Testing for this project will focus on validating both functional performance and usability across the four core components. For the multi-dimensional sprint planner, tests will assess the accuracy and adaptability of task prioritization and scheduling in response to simulated changes in team availability, task dependencies, and deadlines. The Eisenhower Matrix task generator will be evaluated on its ability to categorize tasks correctly based on urgency and importance, as well as its impact on user focus and task completion rates.
The version control system and commit summarization feature will be tested for robustness in tracking file changes, accuracy of LLM-generated summaries, and their usefulness in aiding project recall and progress reporting. Context-switching support will be assessed through user testing to measure reductions in ramp-up time when switching between tasks or projects.
Finally, the containerized deployment will be tested for security, performance, and reliability in an isolated enterprise environment. This includes ensuring data remains local, validating LLM module swap-ins do not disrupt functionality, and confirming that all features work without requiring prompt engineering. Across all components, user feedback will be collected to evaluate usability, clarity, and relevance—ensuring the system meets real-world enterprise expectations.

% 2. Background & Literature Review
\chapter{Background and Literature Review}  % ~2000 words


% 3. Methodology
\chapter{Methodology}  % ~1500 words
\section{Requirements Gathering}
\section{Design Methodology}
\section{Data Sources and Preprocessing}
\section{LLM Selection and Integration Strategy}
\section{System Architecture Overview}

% 4. System Design & Implementation
\chapter{System Design and Implementation}  % ~2000 words
\section{System Architecture}
\section{Frontend and Backend Design}
\section{Key Features and Functionalities}
\section{LLM Integration and Prompt Engineering}
\section{Security and Ethical Considerations}

% 5. Evaluation & Testing
\chapter{Evaluation and Testing}  % ~1500 words
\section{Evaluation Criteria}
\section{Functional Testing}
\section{LLM Response Evaluation}
\section{User Testing and Feedback}
\section{Limitations in Testing}

% 6. Discussion
\chapter{Discussion}  % ~1000 words
\section{Analysis of Results}
\section{Challenges Faced}
\section{Agile Methodology Reflection}
\section{Model Performance vs Expectations}

% 7. Conclusion & Future Work
\chapter{Conclusion and Future Work}  % ~700 words
\section{Summary of Contributions}
\section{Applications and Impact}
\section{Limitations}
\section{Future Improvements}

% References

\printbibliography% \bibliographystyle{plainnat}

% Appendices (not included in word count)
\appendix
\chapter{Detailed Literature Review}
As this software project aims to assist enterprise level organisations with project management, it is vital for it to interface with the companies’ existing project management frameworks. It is  important to ensure that the workflow of the software does not in itself become a barrier to adoption, necessitating software specific training and introducing new inefficiencies as a result. 
In order to ascertain which framework would be the best to start working with, I read the paper “Analysis and Comparison of Project Management Standards and Guides” \parencite{xueAnalysisComparisonProject}. This paper addresses a significant challenge faced by project managers: selecting the most appropriate project management (PM) standard or guide to improve project success across different organisational contexts. Recognising that project management is critical for successful project delivery, the paper sets out to analyse and compare three prominent PM references—PMBoK, ISO 21500, and ISO/IEC 29110—to assist project managers in making an informed choice tailored to their project scale and needs. The central problem tackled is the difficulty project managers encounter in navigating and selecting among various PM standards due to the abundance and complexity of these references.
The motivation behind this study is well grounded in the evolution of project management practices and the increasing need for clear guidance, especially in a landscape where projects vary vastly in size, complexity, and organizational structure. The paper begins by providing a historical context, illustrating that project management has been practised for millennia, and has continuously evolved with tools like Gantt charts, The Critical Path Method (CPM), Work Breakdown Structure (WBS), and Earned Value Management (EVM) becoming integral to contemporary PM. However, the variety of standards and guides—each designed with different target audiences—creates confusion for project managers, especially those unfamiliar with the subtle differences or lacking the time to conduct deep comparative analysis.
The paper’s primary purpose is to demystify this landscape by thoroughly comparing the three most widely recognized PM references:
    1. PMBoK (Project Management Body of Knowledge), published by PMI, is described as the most comprehensive and detailed guide. It defines 47 processes grouped into 10 Knowledge Areas (KAs) and organizes these into five process groups aligned with project phases (initiating, planning, executing, monitoring and controlling, closing). This level of detail includes specific tools and techniques, which support managers in navigating complex project environments, especially large-scale projects. The paper highlights PMBoK’s adaptability across different systems engineering stages and its broad acceptance, which makes it the go-to guide for many professional project managers.
    2. ISO 21500, published by the International Organisation for Standardisation, shares a similar structure to PMBoK with its five process groups and 10 subject groups. Its purpose is to provide guidance that can be applied by any type of organisation, regardless of size. While it has fewer processes (39) than PMBoK, it offers flexibility and international standardisation, making it suitable for organisations looking for a universally recognised framework. The paper notes that ISO 21500 includes descriptions of inputs and outputs for each process but is less prescriptive on tools and techniques compared to PMBoK.
    3. ISO/IEC 29110 is distinctively tailored for Very Small Entities (VSEs)—organizations with fewer than 25 employees. This standard acknowledges the challenges faced by VSEs in applying traditional PM and software engineering standards due to their limited scale, time, and resources. The guide simplifies processes to just two key areas: project management and software implementation. Its modular structure (five parts) allows VSEs to adapt and tailor the guide to their needs, emphasising usability and practicality over comprehensiveness. The paper underlines that this makes ISO/IEC 29110 particularly valuable for small companies acting as suppliers in larger supply chains.
The results of the comparative analysis lead to clear, actionable conclusions aligned with the paper’s objective to support project managers in choosing the right standard. PMBoK is best suited for large, complex projects requiring detailed process guidance and comprehensive tools. ISO 21500 offers a lighter, internationally standardized framework suitable for a wide range of organizations, especially those seeking alignment with global standards. ISO/IEC 29110 addresses a niche yet critical market segment by enabling very small companies to manage projects successfully with an appropriately scaled standard.
This comparison directly responds to the problem statement by reducing the complexity around PM standard selection, thereby facilitating better adoption of PM practices tailored to project and organisational size. Furthermore, the paper’s discussion about the trend of large companies outsourcing to smaller suppliers underscores the growing importance of matching standards to organisational scale, especially as supply chains become more fragmented. 
To Gain a greater understanding, more research was conducted to identify the different project management methodologies used. Reading ‘Analysis of the Available Project Management Methodologies’ \parencite{jovanovicAnalysisAvailableProject2018} once again highlighted the role that project methodologies can play, The effective implementation of project management methodologies remains a critical concern for organisations navigating increasingly complex and diverse project environments. This paper addresses a well-documented challenge in project management literature: the inadequacy of one-size-fits-all methodologies to accommodate the heterogeneous nature of projects across different sectors and scales. The central purpose of this study was to analyse existing project management methodologies—such as PMI, IPMA, PRINCE2, YUPMA, APM, and Agile—and to evaluate their applicability to distinct project groups based on project characteristics and organisational context.
The paper’s problem statement identifies a key gap: while numerous internationally recognised methodologies offer structured frameworks and processes, they often fail to consider the unique attributes of particular project types. This limitation is especially salient given the wide variance in project complexity, scope, and sectoral demands. Consequently, the research objective is to conduct a meta-analysis of extant methodologies, elucidating their specific strengths and weaknesses, with a view toward enabling the tailored selection or development of methodologies suited to project groups with shared characteristics.
Findings from the meta-analysis reveal that traditional, process-centric methodologies such as PMI, PRINCE2, APM, and YUPMA predominantly align with large-scale, complex projects typical of investment, military, manufacturing, and extensive infrastructure undertakings. These methodologies provide comprehensive coverage of project knowledge areas and processes, offering robust governance but often at the expense of flexibility and contextual sensitivity. In contrast, the IPMA methodology’s emphasis on project manager competencies: technical, behavioural, and contextual, shifts the focus from rigid processes to the capabilities of individuals, reflecting a more adaptive but less prescriptive approach.
Agile methodologies emerge from the analysis as particularly well-suited to smaller, less complex projects, especially within the IT sector. Their iterative and flexible nature accommodates rapid change and uncertainty, addressing deficiencies in traditional methodologies when applied to dynamic or fast-paced project environments. However, Xue et. al. finds Agile’s applicability outside IT and similarly scoped projects remains limited.
The study’s synthesis underscores a critical implication: the suitability of any project management methodology is contingent on the interplay between project type, organisational culture, and management philosophy. This reinforces the necessity of moving beyond generalised prescriptions towards developing or selecting methodologies that reflect the specific demands and contexts of project groups. The paper thus positions itself as a foundational step in ongoing research aimed at grouping projects by similarity and tailoring methodologies accordingly.
In relation to the problem statement, the results substantiate the argument that existing methodologies are not universally optimal. They highlight the risk of misapplication and inefficiency when project managers adopt methodologies without considering contextual fit. This resonates with calls in contemporary project management scholarship advocating for greater methodological pluralism and context-driven adaptation. In addition, the on going research hints at a potentially vital feature set for future / further development of the software: a modular system that can interface with a range of different project management processes, ensuring that the correct framework is used for the nature of each project and the team members involved whilst also ensuring that data flows smoothly between different projects following different management protocols. 
A stand out point here is the claim made by Xue et. al. that the efficacy of Agile is limited. Serrador and Pinto explore the tangible impact of Agile methodologies on project success, offering empirical evidence to support the growing adoption of Agile practices in contemporary project management \parencite{serradorDoesAgileWork2015}. The core aim of the study is to assess whether the degree of Agile usage in a project is positively correlated with various dimensions of project success, namely overall success, efficiency (time and budget), and stakeholder satisfaction. This objective is positioned within a broader critique of the historically high failure rates of projects despite adherence to traditional methodologies such as waterfall.
The paper is underpinned by data from 1,386 projects across various industries, providing a substantial empirical base. The methodology quantifies the extent of Agile implementation on a scale from 0% to 100% and measures its correlation with perceived project outcomes. The results reveal a statistically significant, though modest, positive relationship between Agile usage and project success, especially in terms of stakeholder satisfaction and overall success. While efficiency gains are less pronounced, they are still statistically valid. Notably, Agile practices did not reduce planning activity overall; rather, they redistributed it: less upfront, more iterative, suggesting a more adaptive planning model rather than an absence of structure.
Importantly, the study explores moderating variables such as project complexity, team experience, and clarity of project vision. Contrary to expectations, neither complexity nor experience significantly moderated the Agile-success relationship. However, projects with clearer visions saw enhanced benefits from Agile methods, indicating that the methodology’s success is not merely procedural but also context-dependent. The industry dimension adds further nuance: while Agile produced stronger positive effects in tech-oriented, healthcare, and service sectors, it had little to no effect in sectors with rigid planning requirements (e.g., construction, government).
The implications of these findings are directly relevant to this project’s objectives, which involve assessing Agile as a contemporary project management framework and exploring its alignment with project success across varying domains. Serrador and Pinto provide empirical reinforcement for the claim that Agile’s iterative and stakeholder-centric nature aligns more closely with the needs of modern, fast-moving projects. This aligns with the theoretical stance that static, universal methodologies may not be sufficient in today’s complex and dynamic environments.
However, the relatively low explanatory power (R² values around 0.02–0.15) underscores that Agile is not a panacea; other factors contribute significantly to project outcomes. The study advocates for a more sophisticated understanding of Agile, particularly its hybridization with traditional methodologies, an area under explored yet reflective of current industry practice.

To gain a deeper understanding of the role of agile, the study aiming to enhance project efficiency was reviewed. Prica et. al. offers a comprehensive examination of Agile Project Management (Agile PM) as a response to the limitations of traditional project management frameworks in complex, uncertain, and fast-paced environments. Central to this discourse is the critique of hierarchical, command-and-control paradigms, which are increasingly misaligned with the demands of knowledge-based, innovation-driven enterprise operations. This insight resonates directly with the problem statement, which identifies structural inefficiencies rooted in bureaucratic rigidity and human cognitive limits as a primary obstacle in enterprise-scale project execution, despite the widespread adoption of established frameworks like PMBOK and PRINCE2 \parencite{pricaEnhancingProjectEfficiency2025}.
The literature traces the origins of Agile to the Agile Manifesto and Declaration of Interdependence, both of which prioritise adaptability, iterative delivery, and team autonomy over prescriptive process adherence. This decentralised model addresses a key challenge described in the problem statement: the human difficulty in managing high-dimensional, dynamic project data. The Agile emphasis on continuous feedback, sprint-based iterations, and role ownership provides a responsive structure that supports real-time adaptability, a core feature of the Multi-Dimensional Sprint Planner objective. These principles form the foundation for automating scheduling based on fluctuating variables such as team availability and evolving compliance requirements.
The critique of traditional project theory reinforces the urgency for a paradigm shift. Their argument that models such as “management-as-planning” and “thermostat control” are insufficient aligns with the project's focus on replacing rigid planning with adaptive automation. Moreover, Williams et al.’s empirical findings, which show that PMBOK’s traditional frameworks hinder performance under uncertainty, support the necessity for LLM-augmented systems capable of pattern recognition and adaptive task reassignment. These findings validate the need for integrating ML systems that operate in fluid, decentralised contexts without relying on users to manually restructure plans or reports.
Coplien and Harrison’s pattern-based approach to Agile further complements the Eisenhower Matrix-based To-Do List Generator objective. They identify recurring behavioural and structural configurations that promote effective project execution. By learning from these patterns, intelligent systems can be trained to automate task prioritisation not just based on importance or urgency but also in relation to historical project behaviours and human productivity cycles—a layer of complexity that traditional planning tools overlook.
Finally, the literature also engages with Wysocki’s quadrant model, which categorises projects by goal clarity and solution certainty. This framework provides an analytical basis for determining when Agile, adaptive, or even extreme project management techniques are most suitable. The proposed system's ability to shift methodologies dynamically and autonomously mirrors this quadrant-based logic. Wysocki’s model offers useful heuristics for designing a sprint planner that fluidly adapts across quadrants, ensuring that LLM-generated roadmaps remain contextually appropriate.

\chapter{Screenshots and Interface Designs}
\chapter{Test Logs and Scripts}
\chapter{User Feedback and Survey Responses}



\end{document}
